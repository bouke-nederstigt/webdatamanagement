\documentclass[Case_Study.tex]{subfiles} 
\begin{document}

\chapter{Conclusions}

Although this research did not provide a clear proof of the possibility of manipulation of the perception of the participants on an event by using their own voice, we cannot clearly reject the hypothesis. The relationship between the order in which the environments are shown to the participant and the perception on the environments proves that there are more factors of influence. Also the different perceived level of presence in the roller coaster environment as a consequence of the kind of sentences shows us that more factors are involved than expected.

To get more insight into the influential factors on the perception of an environment, more research should be done on this topic. In this research, the exuberance of the audio records of a voice did not match the exuberance of the most extreme environment, the roller coaster. The influence of the exuberance of an environment on the possible change of perception of a participant could be topic for a follow-up research.

In addition, the effect of the type of stimulus could be a topic for follow-up research. In this research, only one's own voice was used as a stimulus. However, there are many more possible stimuli such as sentences from other voices. 

\section{Concluding the research project}
For future research we recommend to use more participants for the pilot-study to check for their reactions on the environment and the plausible emotions people could have during this environment. The results of the pilot-study could be used to find correlations between the answers so questions can be selected or changed for the final questionnaire. In this research the emotion scary, as the opposing emotion for exciting, didn't turn out to give clear results so this measure might be avoided. Boring on the contrary, as the opposing emotion for exciting, did show more relevant results. A larger pilot-study would have made these conclusions clear beforehand.  

\end{document}