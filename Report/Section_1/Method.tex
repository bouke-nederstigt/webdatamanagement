 \documentclass[Method.tex]{subfiles} 

\begin{document}

\chapter{Experiment}
We conducted social surveys, live experiments and pilot studies to study our hypothesis. The conceptual model in Figure \ref{fig:independent_vs_dependent} shows the independent and dependent variables in our experiment.

\begin{figure} [H]
	\centering
	\includegraphics[width=1\textwidth]{./Figures/Independent_vs_Dependent.png}
	\caption{Conceptual model of the experiment}
	\label{fig:independent_vs_dependent}
\end{figure}
% Figure \ref{fig:independent_vs_dependent}

\section{Materials}
The materials used for the experiment were all freely available to us. For recording the audio we used Audacity, an open-source audio editing application, and for audio playback we used VLC Media Player. The virtual reality environments used in the pilot and the experiment stages were also publicly available for use on the internet.\footnote{http://www.riftenabled.com/admin/apps} The Oculus Rift headset (property of TU Delft Interactive Intelligence group) allowed participants to experience the virtual environments. Finally, online surveys were conducted using Google Forms.

\section{Setup}

\subsection{Pilot}
Before we conducted the real experiment with 30 participants, we ran a pilot experiment with 5 participants. The goals for the pilot were:

\begin{itemize}
\item Decide on which virtual environments to use for the real experiment
\item Get feedback on audio stimuli - number of stimuli, content, timing
\item Decide which environments were the most and least impactful on the participants in terms of the following emotions - exciting/scary and exciting/boring
\item Finalize on the time duration of the environments
\end{itemize}

The pilot process lasted about an hour for each participant during which we asked participants for any pre-existing conditions and conducted the experiment with 4 different virtual environments. It was a very extensive process and based on the conclusive results from this experiment, we chose the roller coaster environment with the emotions exciting and scary and the space environment with the emotions exciting and boring as the final choice of environments. The results from the pilot also helped us fine tune any audio issues that we had during the pilot and we incorporated the other feedback from the participants as well.

\subsection{Experiment Process}
For the final experiment, we surveyed 30 volunteers (4 females, 26 males) aged between 18 and 39 years (M = 23.64, SD = 4.262) currently working or studying at the university of Delft. They gave informed written consent and were not paid for their participation. No subject had any experience with the Oculus rift. The study was approved by the local Ethics Committee (University of Delft).

The whole process lasted about 30 minutes per participant and is illustrated below in Figure \ref{fig:experiment_process}.

\begin{figure} [H]
	\centering
	\includegraphics[width=0.4\textwidth]{./Figures/experiment_process.png}
	\caption{Experiment Process for every participant}
	\label{fig:experiment_process}
\end{figure}

Delving deeper into the stimulus recording block from Figure \ref{fig:experiment_process}, each participant recorded two sentences in English with distinct emotions for both environments. This was done to ensure consistency in the emotional states. The emotions for the stimuli were boring and exciting for the space environment and exciting and scary for the roller coaster environment. 

The final stimulus for both the environments was randomly chosen. For the audio recording, we used preset sentences for the participants to record their own voice  at appropriate times for each environment. They were encouraged to be very expressive when they did the recordings in order to get a more realistic experience.

The audio recordings were played back to the participants through headsets while watching the environments using the Oculus Rift 3-D glasses. The environments with the audio were played in random order, according to one of the schemes in Figure \ref{fig:experiment_config}. 

\begin{figure} [H]
	\centering
	\includegraphics[width=1\textwidth]{./Figures/experiment_config.png}
	\caption{Experiment Process Configurations for participants}
	\label{fig:experiment_config}
\end{figure}

Once the participants finished watching the environments with the overlaid audio recordings of their own voice, they were asked to complete a short post survey where we asked them for any mental or physical discomfort. Finally, the participants completed the final questionnaire 3 days after doing the experiment where they were asked more detailed questions regarding both the environments and how they felt before and after the experiences. The results and findings are shown in greater detail in the Section 4 below.


\end{document}
